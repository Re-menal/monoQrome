\documentclass[uplatex,dvipdfmx,a4paper]{jsarticle}

%##
%slug:real-number
%id:1.1
%title:テストページ
%date:2022-04-05
%##

% 
\usepackage{silence}
\WarningFilter{latexfont}{Some font shapes}
\WarningFilter{latexfont}{Font shape}
%

% パッケージ
%%%%%%%%%%%%%%%%%%%%%%%%%%%%%%%%%%%%%%
\usepackage{amsthm,amsmath, mathtools, amssymb, mathrsfs}
% amsthm -> 定理環境。
% amsmath, mathtools -> 基本の数式 amsmath は mathtools に含まれる。
% amssymb -> 記号の追加。 amsfonts は amssymb に含まれる。
% mathrsfs -> 花文字。 rsfso を使うと落ち着いた書体になる(美文書 p90)。
% ※ latexsym は amsmath に含まれるようになった(ただし字形が微妙に異なる)。
\usepackage{framed}
% framed -> 枠。
\usepackage{tikz}
\usetikzlibrary{arrows}
%%%%%%%%%%%%%%%%%%%%%%%%%%%%%%%%%%%%%%

% 定理 & コマンド
%%%%%%%%%%%%%%%%%%%%%%%%%%%%%%%%%%%%%%
\theoremstyle{definition}
\newtheorem{theorem}{定理}[subsection]
\newtheorem{proposition}[theorem]{命題}
\newtheorem{lemma}[theorem]{補題}
\newtheorem{cor}[theorem]{系}
\newtheorem{definition}[theorem]{定義}
\newtheorem{ex}[theorem]{例}
\newtheorem{remark}[theorem]{注意}
\newtheorem*{remark*}{注意}
\newtheorem{pg}[theorem]{ }
\newtheorem*{cons}{考察}
\newtheorem{step}{Step}
\renewcommand{\proofname}{\textbf{証明}}

\newcommand{\N}{\mathbb{N}}
\newcommand{\R}{\mathbb{R}}
\newcommand{\bDelta}{\mathbbl{\Delta}}
\newcommand{\Ob}{\mathrm{Ob}}
\newcommand{\one}{\mathbf{1}}
\newcommand{\id}{\mathrm{id}}
\newcommand{\si}{\scriptstyle}
\newcommand{\dps}{\displaystyle}
\newcommand{\ve}{\varepsilon}
\newcommand{\cl}[1]{\overline{#1}}
%%%%%%%%%%%%%%%%%%%%%%%%%%%%%%%%%%%%%%

%
\title{関数の解析}
\author{Re-menal}
\date{\today}
%

\begin{document}
\begin{definition}
\label{def-limit-of-function}
$A \subset D,\ a \in \cl{A},\ \alpha \in \R$ とする。
\begin{enumerate}
    \item \textbf{$x \to a$ のとき $f(x)$ が $A$ において極限 $\alpha$ に収束する}とは、任意の $\ve > 0$ に対して、ある $\delta > 0$ が存在して、 $|x - a| < \delta$ なる任意の $x \in A$ に対し $|f(x) - \alpha| < \ve$ が成り立つことをいう。
    このとき、記号で 
    \begin{align*}
        & \lim_{x \to a,\ x \in A} f(x) = \alpha \\ 
        & f(x) \to \alpha\ (x \to a,\ x \in A)
    \end{align*}
    などとかく。
    また、 $A = D$ であるとき、単に\textbf{$x \to a$ のとき $f(x)$ が極限 $\alpha$ に収束する}といい、
    \begin{align*}
        & \lim_{x \to a} f(x) = \alpha \\ 
        & f(x) \to \alpha\ (x \to a)
    \end{align*}
    と表す。
    \item \textbf{$x \to a$ のとき $f(x)$ が $A$ において $+\infty$ に発散する}とは、任意の $M > 0$ に対して、ある $\delta > 0$ が存在して、 $|x - a| < \delta$ なる任意の $x \in A$ に対し $M < f(x)$ が成り立つことをいう。
    また、 \textbf{$x \to a$ のとき $f(x)$ が $A$ において $-\infty$ に発散する}とは、任意の $M > 0$ に対して、ある $\delta > 0$ が存在して、 $|x - a| < \delta$ なる任意の $x \in A$ に対し $f(x) < M$ が成り立つことをいう。
\end{enumerate}
\end{definition}
%++
\begin{proposition}
\label{prop-charactarization-of-continuity-of-function}
$a \in D$ に対し、次の 2 条件は同値。
\begin{enumerate}
    \item $f$ は $a$ において連続。
    \item 極限 $\dps \lim_{x \to a} f(x) = f(a)$ が存在する。
\end{enumerate}
\end{proposition}
%--
\begin{proof}
    %Step1
    1 ステップ目。
    \par
    テストテスト。
    \par
    テスト。
    \par
    %Step2
    2 ステップ目。
    \par
    かきくけこ。
    \begin{oframed}%1
        詳細。
        \par
        テスト。
        \par
        テストテスト。
    \end{oframed}%1
    \par
    終了。
\end{proof}
%++
\begin{theorem}
テストの定理。
\begin{enumerate}%1
    \item 次の 2 つが成り立つ。
        \begin{itemize}%2
            \item 主張 1
            \item 主張 2
        \end{itemize}%2
    \item 主張 3
\end{enumerate}%1
加えて、
\begin{itemize}%1
    \item X が次の 2 条件
        \begin{enumerate}%2
            \item 条件 1
            \item 条件 2
        \end{enumerate}%2
    をみたすとき、P が成り立つ。
    \item 主張 Q 。
\end{itemize}%1
\end{theorem}
\end{document}